\documentclass{amsart}

\usepackage{amsmath, amsthm, amscd, amsfonts, amssymb, graphicx, color}
\usepackage{tikz-cd}
% \usepackage{amsmath}
% \usepackage[backend=biber]{biblatex}
% \addbibresource{references.bib}

\newtheorem{teorem}{Teorem}%[section]
\newtheorem{lemma}[teorem]{Lemma}
\theoremstyle{definition}
\newtheorem{definisjon}[teorem]{Definisjon}
\newtheorem{eksempel}[teorem]{Eksempel}
\newtheorem{oppgave}{Oppgave}
\theoremstyle{remark}
\newtheorem{bemerkning}[teorem]{Bemerkning}
\numberwithin{equation}{section}

%\newcommand{\V}[1]{\mathbf{#1}}
\newcommand{\punkt}[1]{\mathbf{#1}}
\newcommand{\vektor}[1]{\vec{\mathbf{#1}}}
\newcommand{\R}{\mathbb{R}}
\newcommand{\F}{\punkt{F}}
\newcommand{\G}{\punkt{G}}
\newcommand{\x}{\punkt{x}}
\newcommand{\y}{\punkt{y}}
\newcommand{\vv}{\vektor{v}}
% \newcommand{matrise}[1]{\begin{bmatrix} #1 \end{bmatrix}}
\begin{document}


\title{Oppgaver Uke 10}
\author{Morten}
\date{\today}
\maketitle

% \section{Kurver}

I dette avsnittet jobber vi med $\mathbb{R}^n$ der $n$ er et positivt heltall. Nesten
alle eksemplene er i $\mathbb{R}^2$ eller $\mathbb{R}^3$.

At $\punkt{y} \in \mathbb{R}^n$ betyr at $\punkt{y}$ er en (kolonne) vektor i $\mathbb{R}^n$.
Det vil si at $\punkt{y}$ er en liste med $n$ reelle tall på formen
$$\punkt{y} = \begin{bmatrix} y_1 \\ y_2 \\ \vdots \\ y_n \end{bmatrix}.$$
Vi kaller $y_i$ for {\em komponentene} til $\punkt{y}$.
Hvis vi vil tenke på $\punkt y$ som en retning skriver vi $\vektor y$ i stedet for $\punkt y$.

\begin{definisjon}
  En {\em parametrisert kurve} i $\mathbb{R}^n$ er en funksjon $\punkt y: I \to
  \mathbb{R}^n$ hvor $I$ er et intervall i $\mathbb{R}$.
\end{definisjon}

At $\punkt y$ er en parametrisert kurve betyr at for hver $t \in I$ har vi en punkt $\punkt y(t) \in \mathbb{R}^n$
som vi kan skrive på formen
$$\punkt y(t) = \begin{bmatrix} y_1(t) \\ y_2(t) \\ \vdots \\ y_n(t) \end{bmatrix}.$$

\begin{eksempel}
  Løsningene vi fant til systemer av differensialligninger er eksempler på parametriserte kurver.
\end{eksempel}

\begin{definisjon}
  En parametrisert kurve $\punkt y: I \to \mathbb   {R}^n $ er {\em deriverbar}
  dersom alle komponentene $y_i(t)$ er deriverbare funksjoner av $t$.
  Den deriverte av $\punkt y$ er den parametriserte kurven
  $$\frac{d \punkt y}{dt}(t) = \begin{bmatrix} y_1'(t) \\ y_2'(t) \\ \vdots \\ y_n'(t) \end{bmatrix}.$$
  Vi kaller $\frac{d \punkt y}{dt}(t)$ for {\em hastighetsvektoren} til kurven $\punkt y$ ved tid $t$.
\end{definisjon}


\begin{eksempel}
  La $\punkt y: [0, 2\pi] \to \mathbb{R}^2$ være gitt ved
  $$\punkt y(t) = \begin{bmatrix} \cos t \\ \sin t \end{bmatrix}.$$
  Dette er en deriverbar parametrisert kurve i $\mathbb{R}^2$ som beskriver en sirkel.
  Den deriverte kurven til $\punkt y$ er
  $$\frac{d \punkt y}{dt} = \begin{bmatrix} -\sin t \\ \cos t \end{bmatrix}.$$
  Bemerk at også $\frac{d \punkt y}{dt}$ er en deriverbar parametrisert kurve som beskriver en sirkel.
\end{eksempel}
[Illustrasjoner]

\begin{eksempel}
  La $\punkt y: [0, 1] \to \mathbb{R}^3$ være gitt ved
  $$\punkt y(t) = \begin{bmatrix} \cos t \\ \sin t \\ t \end{bmatrix}.$$
  Dette er en parametrisert kurve i $\mathbb{R}^3$ som beskriver en spiral.
\end{eksempel}
[Illustrasjoner]

\begin{eksempel}
  La $\punkt p$ være et punkt i $\mathbb{R}^n$ og la $\vektor v$ være en vektor i $\mathbb{R}^n$.
  Da er $\punkt y: \mathbb{R} \to \mathbb{R}^n$ gitt ved
  $$\punkt y(t) = \punkt p + t \vektor v$$
  en parametrisert kurve i $\mathbb{R}^n$.
  Denne kurven beskriver en rett linje gjennom punktet $\punkt p$ med retningsvektor $\vektor v$.
\end{eksempel}
[Illustrasjon $n=2$]




% \section{Funksjoner av flere variable}

En funksjon $f: \mathbb{R}^n \to \mathbb{R}$ tar et punkt $\punkt x \in \mathbb{R}^n$
til et reelt tall $f(\punkt x) \in \mathbb{R}$. 

\subsection{Grafen til en funksjon av to variable}
Funksjoner av to variable er funksjoner på formen $f \colon \R^2 \to \R$. Slike
funksjoner kan illustreres ved å tegne den tilhørende graf.

\begin{eksempel}
  La $f: \mathbb{R}^2 \to \mathbb{R}$ være gitt ved
  $$f(\begin{bmatrix} x_1 \\ x_2 \end{bmatrix}) = x_1^2 + x_2^2.$$
  Dette er en funksjon som tar et punkt $\begin{bmatrix} x_1 \\ x_2
  \end{bmatrix} \in \mathbb{R}^2$ til et reelt tall
  $f(\begin{bmatrix} x_1 \\ x_2 \end{bmatrix}) = x_1^2 + x_2^2.$
\end{eksempel}

\begin{eksempel}
  La $A = [a_1, a_2, \dots, a_n]$ være en rad vektor. 
  Gitt et punkt $\punkt x \in \R^n$ kan vi
  tenke på det som en vektor $\vektor x \in \R^n$, og la $f: \mathbb{R}^n \to \mathbb{R}$ være gitt ved
  $$f(\punkt x) = A\cdot \vektor x.$$
  Dette er en funksjon som tar et punkt $\punkt x \in \mathbb{R}^n$ til tallet
  $f(\punkt x) = A\cdot \vektor x \in \mathbb{R}$.
  (Egentlig er et punkt og en vektor det samme!)

  For $n = 2$ er dette en funksjon $f \colon \R^2 \to \R$, så vi kan tegne grafen til $f$.
  Vi illustrer dette for $A = [1, 2]$.
\end{eksempel}

\begin{eksempel}
  La $f$ være gitt ved
  $$f(\begin{bmatrix} x_1 \\ x_2 \end{bmatrix}) = \sqrt{1 - x_1^2 + x_2^2}.$$
  Denne funksjonen er ikke definert for alle punktene i $\R^2$. Hvis vi skriver 
  $D$ for mengden av punkter $\begin{bmatrix}x_1\\x_2\end{bmatrix}$ i $\R^2$
  med $x_1^2 + x_2^2 \le 0$, da kan vi skrive at $f$ er en funksjon $f \colon D \to \R$.
\end{eksempel}

\begin{eksempel}
  La $f \colon \R^2 \to \R$ være gitt ved
  $$f(\begin{bmatrix} x_1 \\ x_2 \end{bmatrix}) =
  2e^{-((x_1 + 1)^2 + x_2^2)}
  +
  e^{-((x_1 - 2)^2 + x_2^2)}
  .$$
  Vi plotter denne funksjonen over rektanglet $[-2,4] \times [-1.5, 1.5]$
  der $-2 \le x_1 \le 4$ og $-1.5 \le x_2 \le 1.5$.
\end{eksempel}

\subsection{Nivåkurver}
Funksjoner kan også illusteres ved å tegne nivåkurver. En nivåkurve for en
funksjon $f \colon \R^2 \to \R$ er en kurve i planet som består av punkter
der funksjonen $f$ har samme verdi. Slike nivåkurver kan illustreres på grafen til 
funksjonen eller de kan tegnes i planet.
Vi illustrerer dette ved å tegne nivåkurver for funksjonene i eksemplene over.



  

% \section{Derivasjon}
\subsection{Vektorer og totalderiverte}
En {\em vektorfunksjon} er en funksjon som tar en verdi i $\R^n$ og gir en
vektor i $\R^m$. En vektorfunksjon kan skrives som en funksjon
$\punkt{F}:\R^n\to\R^m$ som tar en vektor $\punkt{x}\in\R^n$ og gir en vektor
$\punkt{F}(\punkt{x})\in\R^m$. En vektorfunksjon kan også skrives som en liste av
funksjoner, for eksempel
$$\punkt{F}(\punkt{x}) = \begin{bmatrix} f_1(\punkt{x}) \\ f_2(\punkt{x}) \\ \vdots \\ f_m(\punkt{x}) \end{bmatrix}.$$

\begin{definisjon}
Funksjonen $\punkt{F}$ er (total)deriverbar i punktet $\punkt{x}_0$ hvis det finnes en $m \times n$-matrise $\punkt{F}'(\punkt{x}_0)$ slik at
$$\lim_{\punkt{h}\to 0} \frac{\punkt{F}(\punkt{x}_0+\punkt{h}) - \punkt{F}(\punkt{x}_0) -
\punkt{F}'(\punkt{x}_0) \cdot \punkt{h}}{\|\punkt{h}\|} = 0.$$
\end{definisjon}


Vi går ikke inn på detalj om grenseverdier i flere variabler. Alt vi trenger å
vite er at de har samme regneregler som
grenerverdier i en variabel.
Dette medfører at de generelle derivasjonsreglene for funksjoner av en variabel også gjelder for vektorfunksjoner.

\begin{definisjon}
  {\em Lineærtilnærmingen} til en vektorfunksjon $\punkt{F}$ som er deriverbar i punktet $\punkt{x}_0$ er
  gitt ved
  $$L(\punkt{x}) = \punkt{F}(\punkt{x}_0) + \punkt{F}'(\punkt{x}_0) \cdot (\punkt{x} - \punkt{x}_0).$$
\end{definisjon}

\begin{eksempel}
  La $f \colon \R \to \R$ være en vanlig funksjon av en variabel. Betingelsen over kan skrives:
  $$\lim_{h\to 0} \frac{f(x_0+h) - f(x_0) - f'(x_0) \cdot h}{h} =
  \lim_{h\to 0} \frac{f(x_0+h) - f(x_0)}{h} - f'(x_0) = 0.$$
  Her står altså 
  $$f'(x_0) = \lim_{h\to 0} \frac{f(x_0+h) - f(x_0)}{h}.$$
  Dette er den vanlige definisjonen av den deriverte av en funksjon av en variabel.

  Lineærtilnærmingen til $f$ i punktet $x_0$ er gitt ved
  $$L(x) = f(x_0) + f'(x_0) \cdot (x - x_0).$$
  Grafen til $L(x)$ er tangenten i punktet $x_0$ til grafen til $f$.
\end{eksempel}
[illustrasjon av tangent til en funksjon av en variabel]

\begin{eksempel}
  La $\punkt{y} \colon \R \to \R^2$ være en kurve i planet. Betingelsen over kan igjen skrives:
  $$\lim_{h\to 0} \frac{\punkt{y}(x_0+h) - \punkt{y}(x_0) - \punkt{y}'(x_0) \cdot h}{h} =
  \lim_{h\to 0} \frac{\punkt{y}(x_0+h) - \punkt{y}(x_0)}{h} - \punkt{y}'(x_0) = 0.$$
  Her står altså 
  $$\punkt{y}'(x_0) = \lim_{h\to 0} \frac{\punkt{y}(x_0+h) - \punkt{y}(x_0)}{h}.$$
  Denne grensen kan beregens komponentvis slik at hvis 
  $$\punkt{y}(t) =
  \begin{bmatrix} y_1(t) \\ y_2(t) \end{bmatrix},$$ så er 
  $$\punkt{y}'(t) =
  \begin{bmatrix} y_1'(t) \\ y_2'(t) \end{bmatrix}.$$
  Lineærtilnærmingen til $\punkt{y}$ i punktet $t_0$ er gitt ved
  $$L(t) = \punkt{y}(t_0) + \punkt{y}'(t_0) \cdot (t - t_0).$$
  Dette er en parametrisering av tangenten til kurven i punktet $\punkt{y}(t_0)$.
\end{eksempel}
[Illustrasjon av tangent til en kurve i planet]
\subsection{Gradienten og retningsderiverte}
\begin{definisjon}
  {\em Den retningsderiverte} av en flervariabel funksjon $\punkt{F} \colon \R^n \to
  \R$ i punktet $\punkt{x}_0$ i en retning $\vektor v \in \R^n$ er gitt ved
  $$D_{\vektor v} \punkt{F}(\punkt{x}_0) = \lim_{t\to 0} \frac{\punkt{F}(\punkt{x}_0 + t\vektor v) - \punkt{F}(\punkt{x}_0)}{t}.$$
\end{definisjon}
Gitt $\punkt{x_0} \in \R^n$ og en vektor $\vektor v \in \R^n$, er funksjonen 
$$g(t) = \punkt{x_0} + t\vektor v$$
en parametrisering av den rette linjen gjennom $\punkt{x_0}$ i retning $\vektor v$.
Funksjonen $\punkt{F} \circ g$ er en funksjon av en variabel, gitt ved $(\punkt{F} \circ g)(t) = \punkt{F}(g(t))$.
Den deriverte av denne kompositten
i punktet $t=0$ er retningsderiverte av $\punkt{F}$ i punktet $\punkt{x_0}$ i retning
$\vektor v$. Det vil si at
$D_{\vektor v} \punkt{F}(\punkt{x_0}) = (\punkt{F} \circ g)'(0)$.

Hvis $D_{\vektor v} \punkt{F}(\punkt{x_0}) > 0$ da vokser funksjonen $\punkt{F}$ når vi beveger oss
vekk fra punketet $\punkt{x}_0$ i retning $\vektor v$. Hvis $D_{\vektor v} \punkt{F}(\punkt{x_0}) < 0$ da
avtar funksjonen $\punkt{F}$ når vi beveger oss vekk fra punktet $\punkt{x}_0$ i retning $\vektor v$.

[Illustrasjon]

\begin{oppgave}\label{retningsderivertderivert}
  Forklar hvorfor $D_{\vektor v} \F(\x_0) = \F'(\x_0)\cdot \vektor{v}$.
\end{oppgave}
% Hint: Bruk at 
% $$\lim_{t \to 0} \frac{\F(\x_0 + t \vv) - \F(\x_0) - \F'(\x_0)}{t} = 0$$
  Husk at enhetsvektoren $\vektor e_i$ er en vektor med lengde 1 i retning $i$-aksen. Det vil
  si at $\vektor e_i = \begin{bmatrix} 0 \\ \vdots \\ 1 \\ \vdots \\ 0 \end{bmatrix}$, der 1
  står på $i$-te plass.
\begin{definisjon}
  Den retningsderiverte av $\punkt{F}$ i punktet $\punkt{x}_0$ i retning $\vektor e_i$
  kalles den $i$-te partiellderiverte av $\punkt{F}$ i punktet $\punkt{x}_0$ og skrives
  $$\frac{\partial \punkt{F}}{\partial x_i}(\punkt{x}_0) = D_{\vektor e_i} \punkt{F}(\punkt{x}_0).$$
\end{definisjon}
Fra oppgave \ref{retningsderivertderivert} har vi at
$$\frac{\partial \F}{\partial x_i}(\x_0) = \F'(\x_0)\cdot \vektor{e}_i.$$
\begin{oppgave}
  Per definisjon er den retningsderiverte av en vektorfunksjon den deriverte til en funksjon av en variabel. 
  Beskriv en funksjon $h \colon \R \to \R$ slik at $\frac{\partial \punkt{F}}{\partial x_i}(\punkt{x}_0) = h'(0)$.
\end{oppgave}
% \begin{oppgave}
%   La $A$ og $B$ være to $n \times n$ matriser med $A \cdot \vektor{e}_i = B \cdot \vektor{e}_i$ for $i = 1, \dots, n$. Forklar 
%   hvorfor vi vet at $A = B$.
% \end{oppgave}
% Hint: Bruk at 
% $$\lim_{t \to 0} \frac{\F(\x_0 + t \vv) - \F(\x_0) - \F'(\x_0)}{t} = 0$$
\begin{definisjon}
  {\em Gradienten} til en vektorfunksjon $F \colon \R^n \to \R$ i et punkt $\x_0 \in \R^n$ er vektoren
  $$\nabla \F(\x_0) = \begin{bmatrix}\frac{\partial \F}{\partial x_1}(\x_0) \\ \vdots \\ \frac{\partial \F}{\partial x_n}(\x_0)\end{bmatrix} $$
\end{definisjon}
\begin{bemerkning}
  Gradienten $\nabla \F(\x_0)$ er en kolonnevektor, og $\F'(\x_0)$ er
  radvektoren som fås ved å legge denne kolonnevktoren ned som vi gjorde da vi
  deinerte matrisemultiplikasjon.
\end{bemerkning}
\begin{teorem}
  Hvis $\F\colon \R^n \to \R$ er en skalarfunksjon og $\x_0$ er et punkt slik
  at $\F'(\x_0)$ er definert, så er
  $$D_{\vektor v} \F(\x_0) = \F'(\x_0) \cdot \vv = \nabla \F(\x_0) \cdot
  \vektor v = |\nabla \F(\x_0)| \cos \theta,$$
  for alle vektorer $\vv$ med lengde $1$, der $\theta$ er vinkelen mellom $\vv$ og $\nabla \F(\x_0)$.
  Derfor er $\nabla \F(\x_0)$ retningen der $\F$ vokser raskest i punktet $\x_0$.
\end{teorem}
[[eksempel]]

\subsection{Gradienter og nivåmengder}

\begin{definisjon}
  En {\em nivåkurve} til en funksjon $\F \colon \R^n \to \R$ er en kurve
  $\punkt{r} \colon \R \to \R^n$
  slik at $\F(\punkt{r}(t)) = c$ for en konstant $c$.
\end{definisjon}
\begin{teorem}
  Gitt $\F \colon \R^n \to \R$ og en nivåkurve $\punkt{r} \colon \R \to \R^n$
  til $\F$ og et tall $t$ slik at de deriverte  $\punkt{r}'(t)$ og $\F'(\punkt{r}(t))$
  begge er definert. Da er 
  $$\nabla \F(\punkt{r}(t)) \cdot \punkt{r}'(t) = 0.$$
\end{teorem}
Dette er en direkte konsekvens av kjerneregelen:
\begin{displaymath}
  0 = (\F \circ \punkt{r})'(t) =
  \F'(\punkt{r}(t)) \cdot \punkt{r}'(t) =
  \nabla F(\punkt{r}(t)) \cdot \punkt{r}'(t).
\end{displaymath}

[[eksempler eg fra boken]
\subsection{Jacobimatrisen}



% \printbibliography
\section{Oppgave}
Løs systemet av differensialligninger:
\begin{align*}
  x'(t) &= x(t) + 2y(t)\\
  y'(t) &= 2x(t) - 2y(t)
\end{align*}
for startbetingelsene:
\begin{enumerate}
  \item $x(0) = 1$ og $y(0) = 0$
  \item $x(0) = 0$ og $y(0) = 1$
  \item $x(0) = 1$ og $y(0) = 1$
\end{enumerate}

\section{Oppgave}
MIP oppgave 11.4.1. Bruk koden under til å løse oppgaven numerisk.

\section{Oppgave}
MIP oppgave 11.4.3. Bruk koden under til å løse oppgaven numerisk.

\section{Oppgave}
MIP oppgave 11.3.4 Bruk koden under til å løse oppgaven numerisk. Løs også oppgaven for hånd.

\section{Oppgave}
Løs systemet av differensialligninger:
\begin{align*}
  x' &= 4x + z \\
  y' &= -2x + y \\
  z' &= -2x + z
\end{align*}
for startbetingelsene:
$x(0) = -1$, $y(0) = 0$ og $z(0) = 1$.
ved å følge disse stegene:
La $\vec v_0 = 
\begin{bmatrix}-1\\0\\1\end{bmatrix}$ og la
$A = \begin{bmatrix}4&0&1\\-2&1&0\\-2&0&1\end{bmatrix}.$
\begin{enumerate}
  \item Finn det karakteristiske polynomet $p(t) = \det(A - tI)$.
  \item Finn røttene til det karakteristiske polynomet $p(t)$. Dette er
    egenverdiene $\lambda_1$, $\lambda_2$ og $\lambda_2$ til systemet.
  \item Finn egenvektorene hørende til $\lambda_1$, $\lambda_2$ og $\lambda_3$.
    Det vil si vektorer $\vec w_1$, $\vec w_2$ og $\vec w_3$ slik at $(A
    -\lambda_1 I) \vec w_1 = 0$, $(A -\lambda_2 I) \vec w_2 = 0$ og $(A
    -\lambda_3 I) \vec w_3 = 0$. Disse kan finnes ved å løse lineære
    ligningssystemer.
  \item Sjekk at for alle $x$, $y$ og $z$ er $\vec v(t) = 
    xe^{\lambda_1 t} \vec w_1 + ye^{\lambda_2 t} \vec w_2 + ze^{\lambda_3 t}w_3$ er en
    funksjon med $\vec v't = A \cdot \vec vt + \vec r$ og $\vec v(t_0) = x\vec
    w_1 + y \vec w_2 + z\vec w_3$. (Her er rollen til $x$, $y$ og $z$ en helt
    annen enn i oppgaveformuleringen.)
  \item Finn $x$, $y$ og $z$ slik at $\vec v_0 = x\vec w_1 + y \vec w_2 + z\vec w_3$.
  % \item Plot løsningen på intervallet $[0, 5]$ der du lar $t_0 = 0$.
\end{enumerate}
Hvis du vil kan du bruke koden under til å gjøre stegene.

\end{document}
