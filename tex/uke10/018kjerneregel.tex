\section*{Kjerneregelen}
\begin{teorem}[Kjerneregelen]
  \label{kjerneregelen}
  La $\F \colon \R^n \to \R^m$ være en funksjon som er differensierbar i
  $\punkt{x}_0 \in \R^n$, og la $\G \colon \R^m \to \R^k$ være en funksjon som er
  differensierbar i $\punkt{y}_0 = \F(\punkt{x}_0) \in \R^m$. Da er funksjonen $\G \circ \F \colon \R^n \to
  \R^k$ differensierbar i $\punkt{x}_0$, og
  $$(\G \circ \F)'(\punkt{x}_0) =  \G'(\punkt{y}_0) \cdot  \F'(\punkt{x}_0).$$
\end{teorem}

\begin{eksempel}
  Vi kan bruke denne versjonen av kjerneregelen til å se at $\frac{d}{dx} x^2 =
  2x$. La $\F \colon \R \to \R^2$ være gitt ved $\F(x) = (x, x)$ og $\G \colon
  \R^2 \to \R$ være gitt ved $\G(x, y) = x \cdot y$.
  De deriverte av $\F$ og $\G$ er gitt ved
  $$\F'(x) = \begin{bmatrix} 1 \\ 1 \end{bmatrix} \quad \text{og} \quad \G'(x, y)
  = \begin{bmatrix} y & x \end{bmatrix}.$$
  Da er $\G \circ \F(x) = x^2$,
  og vi kan bruke kjerneregelen til å regne ut den deriverte:
  \begin{align*}
    \frac{d}{dx} x^2 &= (\G \circ \F)'(x) = \G'(x, x) \cdot \F'(x) \\
    &= \begin{bmatrix} x & x \end{bmatrix} \begin{bmatrix} 1 \\ 1 \end{bmatrix} = x + x = 2x.
    \end{align*}
\end{eksempel}

\begin{eksempel}
  Vi kan bruke denne versjonen av kjerneregelen til forklare produktregelen for
  derivasjon.
  La $f \colon \R \to \R$ og $g \colon \R \to \R$ være to funksjoner som er
  differensierbare i $x_0$.
  La $\F \colon \R \to \R^2$ være gitt ved $\F(x) = (f(x), g(x))$ og $\G \colon
  \R^2 \to \R$ være gitt ved $\G(x, y) = x \cdot y$.
  De deriverte av $\F$ og $\G$ er gitt ved
  $$\F'(x) = \begin{bmatrix} f'(x) \\ g'(x) \end{bmatrix} \quad \text{og} \quad \G'(x, y)
  = \begin{bmatrix} y & x \end{bmatrix}.$$
  Da er $\G \circ \F(x) = x^2$,
  og vi kan bruke kjerneregelen til å regne ut den deriverte:
  \begin{align*}
    \frac{d}{dx} x^2 &= (\G \circ \F)'(x) = \G'(f(x), g(x)) \cdot \F'(x) \\
    &= \begin{bmatrix} g(x) & f(x) \end{bmatrix} \begin{bmatrix} f'(x) \\ g'(x)
  \end{bmatrix} = g(x) f'(x) + f(x) g'(x).
    \end{align*}
\end{eksempel}
\subsection{Gradienter og nivåmengder}

\begin{definisjon}
  En {\em nivåkurve} til en funksjon $f \colon \R^n \to \R$ er en kurve
  $\punkt{r} \colon \R \to \R^n$
  slik at $f(\punkt{r}(t)) = c$ for en konstant $c$.
\end{definisjon}
\begin{teorem}
  Gitt $f \colon \R^n \to \R$ og en nivåkurve $\punkt{r} \colon \R \to \R^n$
  til $f$ og et tall $t$ slik at de deriverte  $\punkt{r}'(t)$ og $f'(\punkt{r}(t))$
  begge er definert. Da er 
  $$\nabla f(\punkt{r}(t)) \cdot \punkt{r}'(t) = 0.$$
\end{teorem}
Dette er en direkte konsekvens av Oppgave \ref{retningsderivertderivert} og kjerneregelen:
\begin{displaymath}
  0 = (f \circ \punkt{r})'(t) =
  f'(\punkt{r}(t)) \cdot \punkt{r}'(t) =
  \nabla f(\punkt{r}(t)) \cdot \punkt{r}'(t).
\end{displaymath}

Hvis $\punkt{r} \colon \R \to \R^n$ er en hvilken som helst deriverbar kurve, så er $\nabla f(\punkt{r}(t)) \cdot \punkt{r}'(t)$ den deriverte til $f \circ \punkt{r}$. Det vil si at $\nabla f(\punkt{r}(t))$ er endringsraten til funksjonen $f$ langs kurven $\punkt{r}$.
