\section{Derivasjon}
\subsection{Vektorer og totalderiverte}
En {\em vektorfunksjon} er en funksjon som tar en verdi i $\R^n$ og gir en
vektor i $\R^m$. En vektorfunksjon kan skrives som en funksjon
$\punkt{F}:\R^n\to\R^m$ som tar en vektor $\punkt{x}\in\R^n$ og gir en vektor
$\punkt{F}(\punkt{x})\in\R^m$. En vektorfunksjon kan også skrives som en liste av
funksjoner, for eksempel
$$\punkt{F}(\punkt{x}) = \begin{bmatrix} f_1(\punkt{x}) \\ f_2(\punkt{x}) \\ \vdots \\ f_m(\punkt{x}) \end{bmatrix}.$$

\begin{definisjon}
Funksjonen $\punkt{F}$ er (total)deriverbar i punktet $\punkt{x}_0$ hvis det finnes en $m \times n$-matrise $\punkt{F}'(\punkt{x}_0)$ slik at
$$\lim_{\punkt{h}\to 0} \frac{\punkt{F}(\punkt{x}_0+\punkt{h}) - \punkt{F}(\punkt{x}_0) -
\punkt{F}'(\punkt{x}_0) \cdot \punkt{h}}{\|\punkt{h}\|} = 0.$$
\end{definisjon}


Vi går ikke inn på detalj om grenseverdier i flere variabler. Alt vi trenger å
vite er at de har samme regneregler som
grenerverdier i en variabel.
Dette medfører at de generelle derivasjonsreglene for funksjoner av en variabel også gjelder for vektorfunksjoner.

\begin{definisjon}
  {\em Lineærtilnærmingen} til en vektorfunksjon $\punkt{F}$ som er deriverbar i punktet $\punkt{x}_0$ er
  gitt ved
  $$L(\punkt{x}) = \punkt{F}(\punkt{x}_0) + \punkt{F}'(\punkt{x}_0) \cdot (\punkt{x} - \punkt{x}_0).$$
\end{definisjon}

\begin{eksempel}
  La $f \colon \R \to \R$ være en vanlig funksjon av en variabel. Betingelsen over kan skrives:
  $$\lim_{h\to 0} \frac{f(x_0+h) - f(x_0) - f'(x_0) \cdot h}{h} =
  \lim_{h\to 0} \frac{f(x_0+h) - f(x_0)}{h} - f'(x_0) = 0.$$
  Her står altså 
  $$f'(x_0) = \lim_{h\to 0} \frac{f(x_0+h) - f(x_0)}{h}.$$
  Dette er den vanlige definisjonen av den deriverte av en funksjon av en variabel.

  Lineærtilnærmingen til $f$ i punktet $x_0$ er gitt ved
  $$L(x) = f(x_0) + f'(x_0) \cdot (x - x_0).$$
  Grafen til $L(x)$ er tangenten i punktet $x_0$ til grafen til $f$.
\end{eksempel}
[illustrasjon av tangent til en funksjon av en variabel]

\begin{eksempel}
  La $\punkt{y} \colon \R \to \R^2$ være en kurve i planet. Betingelsen over kan igjen skrives:
  $$\lim_{h\to 0} \frac{\punkt{y}(x_0+h) - \punkt{y}(x_0) - \punkt{y}'(x_0) \cdot h}{h} =
  \lim_{h\to 0} \frac{\punkt{y}(x_0+h) - \punkt{y}(x_0)}{h} - \punkt{y}'(x_0) = 0.$$
  Her står altså 
  $$\punkt{y}'(x_0) = \lim_{h\to 0} \frac{\punkt{y}(x_0+h) - \punkt{y}(x_0)}{h}.$$
  Denne grensen kan beregens komponentvis slik at hvis 
  $$\punkt{y}(t) =
  \begin{bmatrix} y_1(t) \\ y_2(t) \end{bmatrix},$$ så er 
  $$\punkt{y}'(t) =
  \begin{bmatrix} y_1'(t) \\ y_2'(t) \end{bmatrix}.$$
  Lineærtilnærmingen til $\punkt{y}$ i punktet $t_0$ er gitt ved
  $$L(t) = \punkt{y}(t_0) + \punkt{y}'(t_0) \cdot (t - t_0).$$
  Dette er en parametrisering av tangenten til kurven i punktet $\punkt{y}(t_0)$.
\end{eksempel}
[Illustrasjon av tangent til en kurve i planet]
\subsection{Gradienten og retningsderiverte}
\begin{definisjon}
  {\em Den retningsderiverte} av en flervariabel funksjon $\punkt{F} \colon \R^n \to
  \R$ i punktet $\punkt{x}_0$ i en retning $\vektor v \in \R^n$ er gitt ved
  $$D_{\vektor v} \punkt{F}(\punkt{x}_0) = \lim_{t\to 0} \frac{\punkt{F}(\punkt{x}_0 + t\vektor v) - \punkt{F}(\punkt{x}_0)}{t}.$$
\end{definisjon}
Gitt $\punkt{x_0} \in \R^n$ og en vektor $\vektor v \in \R^n$, er funksjonen 
$$g(t) = \punkt{x_0} + t\vektor v$$
en parametrisering av den rette linjen gjennom $\punkt{x_0}$ i retning $\vektor v$.
Funksjonen $\punkt{F} \circ g$ er en funksjon av en variabel, gitt ved $(\punkt{F} \circ g)(t) = \punkt{F}(g(t))$.
Den deriverte av denne kompositten
i punktet $t=0$ er retningsderiverte av $\punkt{F}$ i punktet $\punkt{x_0}$ i retning
$\vektor v$. Det vil si at
$D_{\vektor v} \punkt{F}(\punkt{x_0}) = (\punkt{F} \circ g)'(0)$.

Hvis $D_{\vektor v} \punkt{F}(\punkt{x_0}) > 0$ da vokser funksjonen $\punkt{F}$ når vi beveger oss
vekk fra punketet $\punkt{x}_0$ i retning $\vektor v$. Hvis $D_{\vektor v} \punkt{F}(\punkt{x_0}) < 0$ da
avtar funksjonen $\punkt{F}$ når vi beveger oss vekk fra punktet $\punkt{x}_0$ i retning $\vektor v$.

[Illustrasjon]

\begin{oppgave}\label{retningsderivertderivert}
  Forklar hvorfor $D_{\vektor v} \F(\x_0) = \F'(\x_0)\cdot \vektor{v}$.
\end{oppgave}
% Hint: Bruk at 
% $$\lim_{t \to 0} \frac{\F(\x_0 + t \vv) - \F(\x_0) - \F'(\x_0)}{t} = 0$$
  Husk at enhetsvektoren $\vektor e_i$ er en vektor med lengde 1 i retning $i$-aksen. Det vil
  si at $\vektor e_i = \begin{bmatrix} 0 \\ \vdots \\ 1 \\ \vdots \\ 0 \end{bmatrix}$, der 1
  står på $i$-te plass.
\begin{definisjon}
  Den retningsderiverte av $\punkt{F}$ i punktet $\punkt{x}_0$ i retning $\vektor e_i$
  kalles den $i$-te partiellderiverte av $\punkt{F}$ i punktet $\punkt{x}_0$ og skrives
  $$\frac{\partial \punkt{F}}{\partial x_i}(\punkt{x}_0) = D_{\vektor e_i} \punkt{F}(\punkt{x}_0).$$
\end{definisjon}
Fra oppgave \ref{retningsderivertderivert} har vi at
$$\frac{\partial \F}{\partial x_i}(\x_0) = \F'(\x_0)\cdot \vektor{e}_i.$$
\begin{oppgave}
  Per definisjon er den retningsderiverte av en vektorfunksjon den deriverte til en funksjon av en variabel. 
  Beskriv en funksjon $h \colon \R \to \R$ slik at $\frac{\partial \punkt{F}}{\partial x_i}(\punkt{x}_0) = h'(0)$.
\end{oppgave}
% \begin{oppgave}
%   La $A$ og $B$ være to $n \times n$ matriser med $A \cdot \vektor{e}_i = B \cdot \vektor{e}_i$ for $i = 1, \dots, n$. Forklar 
%   hvorfor vi vet at $A = B$.
% \end{oppgave}
% Hint: Bruk at 
% $$\lim_{t \to 0} \frac{\F(\x_0 + t \vv) - \F(\x_0) - \F'(\x_0)}{t} = 0$$
\begin{definisjon}
  {\em Gradienten} til en vektorfunksjon $F \colon \R^n \to \R$ i et punkt $\x_0 \in \R^n$ er vektoren
  $$\nabla \F(\x_0) = \begin{bmatrix}\frac{\partial \F}{\partial x_1}(\x_0) \\ \vdots \\ \frac{\partial \F}{\partial x_n}(\x_0)\end{bmatrix} $$
\end{definisjon}
\begin{bemerkning}
  Gradienten $\nabla \F(\x_0)$ er en kolonnevektor, og $\F'(\x_0)$ er
  radvektoren som fås ved å legge denne kolonnevktoren ned som vi gjorde da vi
  deinerte matrisemultiplikasjon.
\end{bemerkning}
\begin{teorem}
  Hvis $\F\colon \R^n \to \R$ er en skalarfunksjon og $\x_0$ er et punkt slik
  at $\F'(\x_0)$ er definert, så er
  $$D_{\vektor v} \F(\x_0) = \F'(\x_0) \cdot \vv = \nabla \F(\x_0) \cdot
  \vektor v = |\nabla \F(\x_0)| \cos \theta,$$
  for alle vektorer $\vv$ med lengde $1$, der $\theta$ er vinkelen mellom $\vv$ og $\nabla \F(\x_0)$.
  Derfor er $\nabla \F(\x_0)$ retningen der $\F$ vokser raskest i punktet $\x_0$.
\end{teorem}
[[eksempel]]

\subsection{Gradienter og nivåmengder}

\begin{definisjon}
  En {\em nivåkurve} til en funksjon $\F \colon \R^n \to \R$ er en kurve
  $\punkt{r} \colon \R \to \R^n$
  slik at $\F(\punkt{r}(t)) = c$ for en konstant $c$.
\end{definisjon}
\begin{teorem}
  Gitt $\F \colon \R^n \to \R$ og en nivåkurve $\punkt{r} \colon \R \to \R^n$
  til $\F$ og et tall $t$ slik at de deriverte  $\punkt{r}'(t)$ og $\F'(\punkt{r}(t))$
  begge er definert. Da er 
  $$\nabla \F(\punkt{r}(t)) \cdot \punkt{r}'(t) = 0.$$
\end{teorem}
Dette er en direkte konsekvens av kjerneregelen:
\begin{displaymath}
  0 = (\F \circ \punkt{r})'(t) =
  \F'(\punkt{r}(t)) \cdot \punkt{r}'(t) =
  \nabla F(\punkt{r}(t)) \cdot \punkt{r}'(t).
\end{displaymath}

[[eksempler eg fra boken]
\subsection{Jacobimatrisen}


