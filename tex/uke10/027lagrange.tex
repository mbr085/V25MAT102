Vi ser nå på to funksjoner $f, g \colon \R^n \to \R$. Problemet vi
stiller oss er å finne et minmum for $f$ på mengden $M = \{ \x \in \R^n \mid
g(\x) = 0 \}$. Vi antar at $f$ og $g$ er differensierbare og at $\nabla g(\x)
\ne 0$ for alle $\x \in M$.

I situasjonen over sier vi også at vi søker et minimum for $f$ under betingelsen $g(x) = 0$.

Vi følger nøye strategien fra gradient descent: La $\y \colon \R \to \R^n$ være en
løsning til differensialligningen
$$\y'(t) = - \nabla f(\y(t)) + \lambda(t) \nabla g(\y(t))$$
der $\lambda \colon \R \to \R$ er funksjonen
$$\lambda(t) = \frac{\nabla f(\y(t)) \cdot \nabla g(\y(t))}{\nabla g(\y(t)) \cdot \nabla g(\y(t))}.$$

Fra kjerneregelen har vi at $(g \circ y)'(t) = \nabla g(\y(t)) \cdot \y'(t).$

\begin{oppgave}
  Vis at $(g \circ y)'(t) = 0$ for alle $t$.
\end{oppgave}

Siden $|\nabla f(\y(t)) \cdot \nabla g(\y(t)) \le |\nabla f(\y(t)| |\nabla g(\y(t))||$, ser vi at

\begin{align*}
  (f \circ g)'(t)
  &= \nabla f(\y(t)) \cdot \y'(t)\\
  &= -|\nabla f(\y(t)|^2 +
  \frac{(\nabla f(\y(t)) \cdot \nabla g(\y(t)))^2}
  {(\nabla g(\y(t)) \cdot \nabla g(\y(t)))}
  \le\\
  &= -|\nabla f(\y(t)|^2 +
  \frac{|\nabla f(\y(t))|^2 |\nabla g(\y(t))|^2}
  {|\nabla g(\y(t))|^2} = 0.
\end{align*}

Som for gradient descent ser vi at dersom $\y(t)$ går mot et punkt $\punkt p$ for $t \to \infty$, da er
$(f \circ \y)'(t) \to 0$ for $t \to \infty$. Siden 
$(f \circ \y)'(t) = -| \nabla f(\y(t))|^2 + \lambda(t)\nabla f(\y(t)) \cdot \nabla
g(\y(t)) $ for alle $t$, ser vi at
$-| \nabla f(\punkt p)|^2 + \lambda(t)\nabla f(\punkt p) \cdot \nabla g(\punkt p) =0$.
Setter si inn for $\lambda(t)$ og setter vi $\widetilde \lambda = \lim_{t \infty} \lambda(t)$ 
får vi at dette bare er mulig dersom $\nabla f(\punkt p) = \widetilde \lambda \nabla g(\punkt p)$.

Omvendt, hvis $\punkt p$ er et minimum for $f$ under betingelsen $g(\x) = 0$,
da er $\nabla f(\punkt p) = \widetilde \lambda \nabla g(\punkt p)$ for en
$\widetilde \lambda$ fordi den konstante funksjonen $\y(t) = \punkt p$ er en løsning til differensialligningen
over.

\begin{teorem}
  Hvis $\punkt p$ er et minimum for $f$ under betingelsen $g(\x) = 0$, da
  er vektorene $\nabla f(\punkt p)$ og $\nabla g(\punkt p)$ parallelle. Det vi
  vil si at der finnes et tall $\widetilde \lambda$ slik at $\nabla f(\punkt p)
  = \widetilde
  \lambda \nabla g(\punkt p)$.
\end{teorem}

\begin{eksempel}
  La $f\colon \R^2 \to \R$ være funksjonen
  $$f(\begin{bmatrix}x\\y\end{bmatrix}) = x^2 + 4y^2 -2x + 8y.$$
  Vi vil finne alle lokale minima for $f$ under betingelsen $x + 2y = 7$.

  La $g\colon \R^2 \to \R$ være funksjonen $g(\begin{bmatrix}x\\y\end{bmatrix})
  = x + 2y - 7$.

  Geometrisk sett sier teoremet til Lagrange at tangentene til nivåkurvene til
  $f$ og $g$ må være parallelle. (Egentlig sier det at gradiengene, som står
  vinkelrett på tangentene, må være parallelle, men det er det samme som å si
  at tangentene er parallelle.)

  Gradientene til $f$ og $g$ er
  $$\nabla f(\begin{bmatrix}x\\y\end{bmatrix}) = \begin{bmatrix}2x - 2\\8y + 8\end{bmatrix}$$
  og 
  $$\nabla g(\begin{bmatrix}x\\y\end{bmatrix}) = \begin{bmatrix}1\\2\end{bmatrix}.$$
  Hvis $\begin{bmatrix}x\\y\end{bmatrix}$ er et minimumspunkt for $f$ under betingelsen $g(\begin{bmatrix}x\\y\end{bmatrix}) = 0$, da må det finnes et tall $ \lambda$ slik at
  $$\begin{bmatrix}2x - 2\\8y + 8\end{bmatrix} =  \lambda
  \begin{bmatrix}1\\2\end{bmatrix}.$$
  Dette kan skrives som to likninger.
  $$
  2x - \lambda = 2$$
  og 
  $$
  8y - 2\lambda = -8.$$
  Begingelsen $g(x) = 0$ blir
  $$x + 2y = 7.$$
  Vi har nå tre likninger og tre ukjente.
\end{eksempel}
\begin{oppgave}
  Løs likningssystemet over. Bruk gjerne Gauss-eliminasjon.
\end{oppgave}



\begin{eksempel}
  La $g\colon \mathbb{R}^2 \to \mathbb{R}$ være funksjonen
  $$g(\begin{bmatrix}x\\y\end{bmatrix}) = x^2 + 4y^2 -2x + 8y -27$$
  og la 
  $$f(\begin{bmatrix}x\\y\end{bmatrix}) = x + 2y.$$
  Vi vil minimere $f$ under betingelsen 
  $$g(\begin{bmatrix}x\\y\end{bmatrix}) = 0.$$
  Vi vil
  finne alle lokale minima for $f$ under betingelsen
  $g(\begin{bmatrix}x\\y\end{bmatrix}) = 0$.
  Som før får vi de to likningene
  $$
  2x - \lambda = 2$$
  og 
  $$
  8y - 2\lambda = -8.$$
  Begingelsen $g(x) = 0$ blir nå
  $$x^2 + 4y^2 -2x + 8y -27 = 0$$

  Jeg har lyst at dette skal bli til en ligning av grad to i en ukjent. Derfor skriver jeg
  ved hjelp av de første to likningene
  $$ x = \lambda/2 + 1$$
  og
  $$ y = \lambda/4 - 1.$$
  Dette setter jeg inn i den andre ligningen:
  \begin{align*}
    0 =
    &x^2 + 4y^2 -2x + 8y -27 \\
    =&
    (\lambda/2 + 1)^2 + 4(\lambda/4 - 1)^2 -2(\lambda/2 + 1) + 8(\lambda/4 - 1) -27\\
    =&
    \lambda^2/4 + \lambda + 1 + \lambda^2/4 - 2\lambda + 4 -\lambda - 2 + 2\lambda - 8 -27\\
    =&
    \lambda^2/2 - 32.
  \end{align*}
  Her står $\lambda^2 = 64 = 8^2$. Derfor er $\lambda = \pm 8$.
  
  Setter jeg inn $\lambda = 8$ får jeg at $x = 5$ og $y = 1$. Setter jeg inn
  $\lambda = -8$ får jeg at $x = -3$ og $y = -3$.

  Siden 
  $$f(\begin{bmatrix}5\\1\end{bmatrix}) = 5^2 + 4 - 2\cdot 5 + 8 = 27$$
  og
  $$f(\begin{bmatrix}-3\\-3\end{bmatrix}) = 9 + 12 + 6 - 24 = 3$$
  må $f(\begin{bmatrix}-3\\-3\end{bmatrix})$ være et minimum for $f$ under
  begingelsen $g = 0$. Altså er punktet 
  $\punkt p = \begin{bmatrix}-3\\-3\end{bmatrix}$ et minimum for $f$ under betingelsen
  $g(\punkt p) = 0$.

\end{eksempel}
