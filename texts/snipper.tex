\documentclass{article}
\usepackage{amsmath}
\begin{document}
Vi har sett på systemer av førsteordens differensialligninger på formen
$$
\begin{align}
  x'(t) &= f(t,x(t), y(t)), \newline
  y'(t) &= g(t,x(t), y(t))
\end{align}
$$
for funksjoner
$$
\begin{align}
\mathbb{R}^3 &\xrightarrow f \mathbb{R}, \newline
(t, x, y) &\mapsto f(t, x, y)
\end{align}
$$
og
$$
\begin{align}
\mathbb{R}^3 &\xrightarrow g \mathbb{R}, \newline
(t, x) &\mapsto g(t, x, y).
\end{align}
$$

Vi kan tenke på dette som en ligning mellom vektorer i $\mathbb{R}^2$. 

Lar vi $\vec v(t) = (x(t), y(t))$ og $F(t, \vec v(t)) = (f(t, x(t), y(t)), g(t, x(t), y(t)))$, kan vi skrive systemet som
$$
\begin{align}
  \vec v'(t) &= F(t, \vec v(t)).
\end{align}
$$
Dette er en kompakt måte å skrive systemet på, og vi kan bruke vektorregning for å løse det.

Hvis vi lar 
$A = \begin{bmatrix}p&q\\r&s\end{bmatrix}$ så kan dette skrives 
$$\vec v'(t) = 
\begin{bmatrix}x'(t)\\y'(t)\end{bmatrix}
= \begin{bmatrix}p&q\\r&s\end{bmatrix} \cdot \begin{bmatrix}x(t)\\y(t)\end{bmatrix}
= A \cdot \begin{bmatrix}x(t)\\y(t)\end{bmatrix}
= A \cdot \vec v(t) = F(t, \vec v(t))$$
Skriver vi 
$\vec v(t) = \begin{bmatrix}x(t)\\y(t)\end{bmatrix}$
er den koordinatvis deriverte til $\vec v(t)$ gitt ved formelen
$\vec v'(t) = \begin{bmatrix}x'(t)\\y'(t)\end{bmatrix}$, og systemet av differensialligninger kan skrives
$$\vec v'(t) = A \cdot \vec v(t).$$

\end{document}

